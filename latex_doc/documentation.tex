\documentclass[10pt,a4paper]{scrartcl}
%=============================================================================
\usepackage[T1]{fontenc}  
\usepackage[utf8x]{inputenc}  
\usepackage[normalem]{ulem} % \emph should italicize, not underline
\usepackage{cite}
\usepackage{graphicx}
%\usepackage{parskip}
\usepackage{xspace}
\usepackage{alltt}
\usepackage{amssymb,textcomp}
\usepackage{ifthen}
\usepackage{amsmath}
\usepackage{amssymb}
\usepackage{subfigure}
\usepackage{amsmath}
\usepackage{xspace}
\usepackage{bold-extra}

\usepackage{needspace}
\newcommand{\needlines}[1]{\Needspace{#1\baselineskip}}

% source code formatting
\usepackage{listings}
	% global settings for source code listing pacakage
\lstset{
    basicstyle=\ttfamily,
    showspaces=false,
    showstringspaces=false,
    captionpos=b, 
    columns=fullflexible}
	% define the listing shortcuts for java and python
\lstnewenvironment{terminalcode}[1][]{\lstset{language=bash,#1}}{} 

%----------------------------------------------------------------------------

% enabled links in pdf, but paint them normal in black
\usepackage[pdftex=true, colorlinks=true, urlcolor=black, 
			linkcolor=black,pagecolor=black,citecolor=black,
			bookmarks=true]{hyperref}

%=============================================================================

\date{HS 2009 Universität Bern}
\author{Camillo Bruni}
\title{Programming Introduction}
\begin{document}

%=============================================================================
\section{Terminal}
%=============================================================================
\subsection{Introduction}
\begin{terminalcode}
> uname -mns
  Darwin imac.local i386
  Report bugs to <bug-coreutils@gnu.org>.
> uname -mns
  Darwin mbkp.local i386
> ssh anker.unibe.ch
  user@bender.unibe.ch's password: 
> uname
  Linux
> uname -mon
  bender x86_64 GNU/Linux
> uname --help
  Usage: uname [OPTION]...
  Print certain system information.  With no OPTION, same as -s.
  
    -a, --all                print all information, in the following order,
                               except omit -p and -i if unknown:
    -s, --kernel-name        print the kernel name
    -n, --nodename           print the network node hostname
    -r, --kernel-release     print the kernel release
    -v, --kernel-version     print the kernel version
    -m, --machine            print the machine hardware name
    -p, --processor          print the processor type or "unknown"
    -i, --hardware-platform  print the hardware platform or "unknown"
    -o, --operating-system   print the operating system
        --help     display this help and exit
        --version  output version information and exit
\end{terminalcode}

%=============================================================================
\subsection{Commands}
\begin{description}
\item[\texttt{cat}]

\item[\texttt{cd}]

\item[\texttt{cp}]

\item[\texttt{ls}]

\item[\texttt{man}]

\item[\texttt{mkdir}]

\item[\texttt{mv}]

\item[\texttt{pwd}]

\item[\texttt{rm}]
    \begin{terminalcode}
cami@bender:~/test$ ls
todelete.txt
cami@bender:~/test$ rm todelete.txt 
cami@bender:~/test$ ls
    \end{terminalcode}

\item[\texttt{ssh}]

\item[\texttt{svn}]

\item[\texttt{touch}] updates the access and modification times of each FILE to 
    the current time.
    \begin{terminalcode}
cami@bender:~/test$ ls -l
-rw-r--r-- 1 cami cami 0 2009-08-25 20:29 date.txt
cami@bender:~/test$ touch date.txt 
cami@bender:~/test$ ls -l
-rw-r--r-- 1 cami cami 0 2009-08-25 20:30 date.txt
    \end{terminalcode}

    It can be very useful to create a new empty file on the fly:
    \begin{terminalcode}
~/test$ ls
~/test$ touch emptyfile.txt
~/test$ ls
emptyfile.txt
    \end{terminalcode}
\end{description}

%=============================================================================
\subsection{Pipes}

\texttt{ls -a | grep log}


1. Terminal Einfuehrung
- kurze Einführung in Bash
- Grundlegende Befehle zur Navigation: ls, cd, pwd und tab completion.

3. Terminal PRO
- Argumente fuer Befehle (ls -a ...)
- Sich selber helfen mit "man"
- Weitere Befehle mit praktischen Übungen: cp, mv, cat (find locate)
- Vielleicht: Enführung in Pipes und grep

%=============================================================================
\section{Documentation with Latex}
%=============================================================================
\subsection{Introduction}
% give general introduction here

%=============================================================================
\subsection{}
% write down most used commands
2. Dokumentation / Latex (Muss am Schluss abgegeben werden und wird  kontrolliert)
- svn co gegebene Projektstruktur
- Kurze Einführung in Latex
- Latex to PDF via vorhandenes Make Script.
- Vorlage bearbeiten und eigene Dokumentation einfügen (copy paste)
- Vielleicht Grundlagen Terminaleditor (Vim oder Emacs) steht noch  zur Diskussion


%=============================================================================
\section{Editing Files with VIM}
%=============================================================================
\subsection{Introduction}

%=============================================================================
\subsection{Basics}
Follow the instruction of the \texttt{vimtutor}.



%=============================================================================
\section{Programming with Ruby}
%=============================================================================
4. Scriptprogrammieren mit Ruby
- Kurze Einführung in Ruby
- Hilfe holen mit Ruby
- Kleine Anweisungen ausführen (ausgabe, loops)
- Anweisungen als ausführbares Script speichern und bearbeiten

5. Erweitertes Programmieren mit Ruby
- Grundlegende Kontrollstrukturen: if und loops
- Werte speichern in Variabeln
- Argumente einlesen
- Bestehende Bibliotheken verwenden
%=======================================================================


\end{document}
