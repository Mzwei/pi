\documentclass[10pt,a4paper]{scrartcl}
%=============================================================================
\usepackage[utf8]{inputenc}  
\usepackage[normalem]{ulem} % \emph should italicize, not underline
\usepackage{alltt}
\usepackage{amsmath}
\usepackage{amssymb}
\usepackage{bold-extra}
\usepackage{cite}
\usepackage{graphicx}
\usepackage{ifthen}
\usepackage{subfigure}
\usepackage{xspace}

% this cane be used to make sure a block of code is printed as whole and not
% split at a page break
\usepackage{needspace}
\newcommand{\needlines}[1]{\Needspace{#1\baselineskip}}

% source code formatting
\usepackage{listings}
	% global settings for source code listing pacakage
\lstset{
    basicstyle=\ttfamily,
    showspaces=false,
    showstringspaces=false,
    captionpos=b, 
    columns=fullflexible}
	% define the listing shortcuts for java and python
\lstnewenvironment{terminalcode}[1][]{\lstset{language=bash,#1}}{} 

%----------------------------------------------------------------------------

% enabled links in pdf, but paint them normal in black
\usepackage[pdftex=true, colorlinks=true, urlcolor=black, 
			linkcolor=black,pagecolor=black,citecolor=black,
			bookmarks=true]{hyperref}

%=============================================================================

\date{HS 2009 Universität Bern}
\author{Camillo Bruni}
\title{Programming Introduction}
\begin{document}

%=============================================================================
\section{Terminal}
%=============================================================================
\subsection{Introduction}
\begin{terminalcode}
> uname -mns
  Darwin imac.local i386
  Report bugs to <bug-coreutils@gnu.org>.
> uname -mns
  Darwin mbkp.local i386
> ssh anker.unibe.ch
  user@bender.unibe.ch's password: 
> uname
  Linux
> uname -mon
  bender x86_64 GNU/Linux
> uname --help
  Usage: uname [OPTION]...
  Print certain system information.  With no OPTION, same as -s.
  
    -a, --all                print all information, in the following order,
                               except omit -p and -i if unknown:
    -s, --kernel-name        print the kernel name
    -n, --nodename           print the network node hostname
    -r, --kernel-release     print the kernel release
    -v, --kernel-version     print the kernel version
    -m, --machine            print the machine hardware name
    -p, --processor          print the processor type or "unknown"
    -i, --hardware-platform  print the hardware platform or "unknown"
    -o, --operating-system   print the operating system
        --help     display this help and exit
        --version  output version information and exit
\end{terminalcode}

%=============================================================================
\subsection{Commands}
\begin{description}

\item[\texttt{rm}]
    \begin{terminalcode}
cami@bender:~/test$ ls
todelete.txt
cami@bender:~/test$ rm todelete.txt 
cami@bender:~/test$ ls
    \end{terminalcode}

\item[\texttt{touch}] updates the access and modification times of each FILE to 
    the current time.
    \begin{terminalcode}
cami@bender:~/test$ ls -l
-rw-r--r-- 1 cami cami 0 2009-08-25 20:29 date.txt
cami@bender:~/test$ touch date.txt 
cami@bender:~/test$ ls -l
-rw-r--r-- 1 cami cami 0 2009-08-25 20:30 date.txt
    \end{terminalcode}

    It can be very useful to create a new empty file on the fly:
    \begin{terminalcode}
~/test$ ls
~/test$ touch emptyfile.txt
~/test$ ls
emptyfile.txt
    \end{terminalcode}


% add your own remarks here by reusing the existing examples

\end{description}

%=============================================================================
\section{Documentation with Latex}
%=============================================================================
\subsection{Introduction}
% give general introduction here

Whenever you need to lookup a certain symbol for \LaTeX\xspace we suggest you to use
the online recognition tool \texttt{detexify} at \url{http://detexify.kirelabs.org/}.


%=============================================================================
\subsection{Common Commands}


%=============================================================================
\section{Ruby Programming}
%=============================================================================


\end{document}
