\documentclass[10pt,a4paper]{scrartcl}
%=============================================================================
\usepackage[utf8]{inputenc}  
\usepackage[normalem]{ulem} % \emph should italicize, not underline
\usepackage{alltt}
\usepackage{amsmath}
\usepackage{amssymb}
%\usepackage{bold-extra}
\usepackage{cite}
\usepackage{graphicx}
\usepackage{ifthen}
\usepackage{subfigure}
\usepackage{xspace}

% source code formatting
\usepackage{listings}
	% global settings for source code listing pacakage
\lstset{
    basicstyle=\ttfamily,
    showspaces=false,
    showstringspaces=false,
    captionpos=b, 
    columns=fullflexible}
	% define the listing shortcuts for java and python
\lstnewenvironment{terminalcode}[1][]{\lstset{language=bash,#1}}{} 

%----------------------------------------------------------------------------

% enabled links in pdf, but paint them normal in black
\usepackage[pdftex=true, colorlinks=true, urlcolor=black, 
			linkcolor=black,pagecolor=black,citecolor=black,
			bookmarks=true]{hyperref}

%=============================================================================

\date{HS 2009 University Bern}
\author{Tobias Brog}
\title{Documentation}
\begin{document}
\maketitle
%=============================================================================
\section{Terminal}
%=============================================================================
\subsection{Introduction}
\begin{terminalcode}
> uname -mns
  Darwin imac.local i386
  Report bugs to <bug-coreutils@gnu.org>.
> uname -mns
  Darwin mbkp.local i386
> ssh anker.unibe.ch
  user@bender.unibe.ch's password: 
> uname
  Linux
> uname -mon
  bender x86_64 GNU/Linux
> uname --help
  Usage: uname [OPTION]...
  Print certain system information.  With no OPTION, same as -s.
  
    -a, --all                print all information, in the following order,
                               except omit -p and -i if unknown:
    -s, --kernel-name        print the kernel name
    -n, --nodename           print the network node hostname
    -r, --kernel-release     print the kernel release
    -v, --kernel-version     print the kernel version
    -m, --machine            print the machine hardware name
    -p, --processor          print the processor type or "unknown"
    -i, --hardware-platform  print the hardware platform or "unknown"
    -o, --operating-system   print the operating system
        --help     display this help and exit
        --version  output version information and exit
\end{terminalcode}

%=============================================================================
\subsection{Commands}
\begin{description}

\item[\texttt{rm}]
    \begin{terminalcode}
cami@bender:~/test$ ls
todelete.txt
cami@bender:~/test$ rm todelete.txt 
cami@bender:~/test$ ls
    \end{terminalcode}

\item[\texttt{touch}] updates the access and modification times of each FILE to 
    the current time.
   	\begin{terminalcode}
cami@bender:~/test$ ls -l
-rw-r--r-- 1 cami cami 0 2009-08-25 20:29 date.txt
cami@bender:~/test$ touch date.txt 
cami@bender:~/test$ ls -l
-rw-r--r-- 1 cami cami 0 2009-08-25 20:30 date.txt
    \end{terminalcode}

    It can be very useful to create a new empty file on the fly:
    \begin{terminalcode}
~/test$ ls
~/test$ touch emptyfile.txt
~/test$ ls
emptyfile.txt
    \end{terminalcode}

% add your own remarks here by reusing the existing examples

\end{description}

%=============================================================================
\section{Documentation with Latex}
%=============================================================================



\begin{itemize}

\item \texttt{\bf to open a file in Vim put "vim" in front of the file's name.}
\item \texttt{"cd" change directory options are -P and -L }
\item \texttt{"ls" list all items in directory. Lists all files in parent directory when you: ls ..
option -R lists subdirectories recursively }
\item \texttt{"wq!" save and close (external command) }
\item \texttt{"w" save}
\item \texttt{".." parent folder . represents current folder}
\item \texttt{\bf "make view" generates a .pdf file in Latex}
\item \texttt{"man" displays the online manual pages for \bf name}
\item \texttt{"pwd" print working directory. The options for pwd are --help and --version.}
\item \texttt{"rm" Remove files or directories Problem:Could not remove directory Solution: use -r option use -f option to avoid prompting. Useful when removing files recursively}
\item \texttt{"mkdir" Make direction.  Could not yet figure out how to create parent directories.}
\item \texttt{"touch" Update the access and modification time to the current time. Use option -t to set time manually in 
CCYYMMDD.ss format }.
\item \texttt{'mv" Write like: mv SOURCE DIRECTORY useful option -u move only when source file is newer than destination file or when destination file is missing. With mv a file can be renamed if destination does not exist. }
\item \texttt{"cp" write like mv command. Similar options to mv. Use -r as with rm. }
\item \texttt{"cat" shows contents of file. Less does the same but on a page. Redirect standard output with > double it without deleting previous >>}
\item \texttt{"ssh" log into a remote machine and execute commands there.Use ~. to disconnect.}
\item \texttt{"svn" Works with a data bin on a server.svn update (up)  - svn commit (ci) uploads changes to repository - svn checkout (co ) gets files from repository - svn stat gets versions svn log gets log entries for designated file}
\item \texttt{"grep" Searches for a PATTERN in given FILES. "grep PATTERN FILE" Prints the matching line With option -v it is possible to invert the sense of matching printing only non matching lines.}
\item \texttt{"less" Displays output on a page. /PATTERN searches and highlights matching patterns. With n and N it is possible to go from one match to the other.}
\item \texttt{"sort" write sorted concatenation of all files to standard output. (Redirection possible too?)}
\item \texttt{"wc" Print newline, byte and word counts for each file}
\item \texttt{"du" Summarize disk usage for each file. For directories recursively.}
\item \texttt{"find" Search a given directory for the provided keyword: find WORD DIRECTORY}
\item \texttt{"wget" Non interactive-download. Does not require the users presence all the time.}
\item \texttt{pipes "|" send information from one program to another. Can also send info from one command to another} 
\item \texttt{ ";" seperates two commands in a line. With double shift-7 between one command and the other the following command gets executed only if command 1 was successful.}
\item \texttt{"svn" update  - svn commit uploads changes to repository - svn checkout gets files from repository - svn stats gets versions}

\item \texttt{For svn ci to work you need to set the log editor: "$export SVN_EDITOR=/usr/bin/vim$}
\end{itemize}

\subsection{Introduction} 

In this section we explain some \LaTeX\xspace details and different formatting
commands.

Whenever you need to lookup a certain symbol for \LaTeX\xspace we suggest you to use
the online recognition tool \texttt{detexify} at \url{http://detexify.kirelabs.org/}.


%=============================================================================
\subsection{Common Commands}
\subsubsection{Sectioning}
Depending on the document class given in the very beginning of this file there
exist several sectioning levels:
\begin{enumerate}
	\item{} \verb$\section{NAME}$
	\item{} \verb$\subsection{NAME}$
	\item{} \verb$\subsubsection{NAME}$
	\item{} \verb$\paragraph{NAME}$
\end{enumerate}

\noindent To enforce \LaTeX to use a newline add a double slash \verb$\\$ at 
the end of a line.

\subsubsection{Schriftgrösse / -style}
\begin{tabular}{lll}                                                          
\verb$\rm$			& {\rm A normaler text}\\ 
\verb$\sl$ 			& {\sl An italic text}\\
\verb$\bf$ 			& {\bf A bold text}\\
\verb$\tiny$ 		& {\tiny A tiny ext}\\
\verb$\scriptsize$ 	& {\scriptsize A very, very small text}\\
\verb$\footnotesize$& {\footnotesize A very small text}\\
\verb$\small$ 		& {\small A small text}\\
\verb$\large$ 		& {\large A big text}\\
\verb$\Large$ 		& {\Large A bigger text}\\
\verb$\LARGE$ 		& {\LARGE An even bigger text}\\
\verb$\huge$ 	    & {\huge A huge text}\\
\verb$\Huge$ 	    & {\Huge A enormous huge text}\\
\verb$\emph$ 	    & \emph{An emphasized text} \\
\verb$\underline$ 	& \underline{An underlined text} \uline{and here using the ulem-package}\\
\verb$\texttt$ 		& \texttt{function goto(int a) { ... } }\\
\verb$\uuline$ 		& \uuline{A double unterstrichener text using the ulem-package} \\
\verb$\uwave$ 		& \uwave{A wavy unterstrichener text using the ulem-package} \\
\verb$\sout$ 	    & \sout{A crossed trough text using the ulem-package}\\
\verb$\xout$ 	    & \xout{A deleted text using the ulem-package}\\
\end{tabular}

\subsubsection{Notes}
To create a footnote use the \verb$\footnote{YOUR NOTE}$ 
command\footnote{\dots as you can see here.}. \\
If you want to put a remark at side of a page use \verb$\marginpar$.
\marginpar{This is a note at the border of the page.}

\subsubsection{Lists}
There exist several list types in \LaTeX. You start a list by adding a 
\verb$\being{LISTTYPE}$ and end it with an \verb$\end{LISTTYPE}$. A list item
is added with a \verb$\item$ between the \texttt{begin} and \texttt{end}.
\texttt{LISTTYPE} can be one of the following list:
\begin{itemize}
	\item \texttt{enumerate}
	\item \texttt{itemize}
	\item \texttt{description} with \verb$\item[topic]$
\end{itemize}
% By adding a \noindet the next line is not indented ;)
\noindent Note that you can nest lists if you want to.
\begin{enumerate}
	\item{e4} 	
		\begin{enumerate}
			\item{e4}   e5
			\item Lc4 d6
		\end{enumerate}
	\item Lc4 d6
\end{enumerate}


%=============================================================================
\section{Ruby Programming}
%=============================================================================

\begin{itemize}

\item \texttt{http://www.ruby-lang.org/en/documentation}
\item \texttt{Easy way to make a String: write  ANYWORD = <<DOC then Enter text as you like. Then end by entering DOC. ANYWORD will show your text.} 

\end{itemize}
\end{document}
