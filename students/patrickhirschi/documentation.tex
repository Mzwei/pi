\documentclass[10pt,a4paper]{scrartcl}
%=============================================================================
\usepackage[utf8]{inputenc}  
\usepackage[normalem]{ulem} % \emph should italicize, not underline
\usepackage{alltt}
\usepackage{amsmath}
\usepackage{amssymb}
%\usepackage{bold-extra}
\usepackage{cite}
\usepackage{graphicx}
\usepackage{ifthen}
\usepackage{subfigure}
\usepackage{xspace}

% source code formatting
\usepackage{listings}
	% global settings for source code listing pacakage
\lstset{
    basicstyle=\ttfamily,
    showspaces=false,
    showstringspaces=false,
    captionpos=b, 
    columns=fullflexible}
	% define the listing shortcuts for java and python
\lstnewenvironment{terminalcode}[1][]{\lstset{language=bash,#1}}{} 

%----------------------------------------------------------------------------

% enabled links in pdf, but paint them normal in black
\usepackage[pdftex=true, colorlinks=true, urlcolor=black, 
			linkcolor=black,pagecolor=black,citecolor=black,
			bookmarks=true]{hyperref}

%=============================================================================

\date{HS 2009 University Bern}
\author{Patrick Hirschi}
\title{Programming Introduction}
\begin{document}
\maketitle
\tableofcontents
\newpage
%=============================================================================
\section{Terminal}
%=============================================================================
\subsection{Introduction}
\begin{terminalcode}
> uname -mns
  Darwin imac.local i386
  Report bugs to <bug-coreutils@gnu.org>.
> uname -mns
  Darwin mbkp.local i386
> ssh anker.unibe.ch
  user@bender.unibe.ch's password: 
> uname
  Linux
> uname -mon
  bender x86_64 GNU/Linux
> uname --help
  Usage: uname [OPTION]...
  Print certain system information.  With no OPTION, same as -s.
  
    -a, --all                print all information, in the following order,
                               except omit -p and -i if unknown:
    -s, --kernel-name        print the kernel name
    -n, --nodename           print the network node hostname
    -r, --kernel-release     print the kernel release
    -v, --kernel-version     print the kernel version
    -m, --machine            print the machine hardware name
    -p, --processor          print the processor type or "unknown"
    -i, --hardware-platform  print the hardware platform or "unknown"
    -o, --operating-system   print the operating system
        --help     display this help and exit
        --version  output version information and exit
\end{terminalcode}

%=============================================================================
\subsection{Commands}
\begin{description}

\item[\texttt{cd}]change directory
    \begin{terminalcode}
Desktop  Documents  Download  Music  Pictures  Public  Templates  tutor  Videos  vimrc
[kurs02@constable ~]$ cd Documents
[kurs02@constable ~/Documents]$ 
    \end{terminalcode}

\item[\texttt{:w}]for saving the document

\item[\texttt{:wq}]for saving the document and exit vim

\item[\texttt{y}]for copying text

\item[\texttt{p}]for pasting text

\item[\texttt{v}]for marking text

\item[\texttt{:q}]quit the program

\item[\texttt{dd}]for deleting the selected line

\item[\texttt{dw}]for deleting the selected word

\item[\texttt{u}]undo the previous action

\item[\texttt{U}]return the line to its original state

\item[\texttt{x}]delete the character under the cursor

\item[\texttt{i}]insert text

\item[\texttt{a}]append text after the cursor

\item[\texttt{A}]append text after the end of the line

\item[\texttt{TAB}]auto tab-completion

\item[\texttt{r}]replace text under the cursor

\item[\texttt{ce}]change till the end of a word

\item[\texttt{G}]move to the bottom of the file

\item[\texttt{Nr. G}]move to the line number

\item[\texttt{gg}]move to the top of the file

\item[\texttt{o}]open a line below the cursor

\item[\texttt{O}]open a line above the cursor

\item[\texttt{0}]move to the start of the line

\item[\texttt{e}]move to the end of the line

\item[\texttt{:make view}]generates a pdf file

\item[\texttt{mkdir NAME}]generates a folder

\item[\texttt{svn up}]update the versions

\item[\texttt{svn checkout}]get the newest version of the document

\item[\texttt{svn add}]add a file to the repository

\item[\texttt{svn status}]watch the status of the files in the repository

\item[\texttt{svn log}]watch the comments on every version of your document

\item[\texttt{svn cat -r versionnumber filename}]open a specific version of the document

\item[\texttt{svn commit -m 'changelog'}]commits the modified documents to the server

\item[\texttt{svn diff -r versionnumber filename}]shows the difference between the selected version and the newest version of the document

\item[\texttt{return}]returns to the file

\item[\texttt{ctrl+c}]interrupt a process

\item[\texttt{ls}]list directory contents
    \begin{terminalcode}
[kurs02@constable ~/Documents]$ ls
documentation  lessons
[kurs02@constable ~/Documents]$ 
    \end{terminalcode}

\item[\texttt{pwd}]print working directory
    \begin{terminalcode}
[kurs02@constable lessons]$ pwd
/home/kurs02/Documents/lessons
[kurs02@constable lessons]$
    \end{terminalcode}

\item[\texttt{man}]get informations about a command

\item[\texttt{rm}] remove directory entries
    \begin{terminalcode}
cami@bender:~/test$ ls
todelete.txt
cami@bender:~/test$ rm todelete.txt 
cami@bender:~/test$ ls
    \end{terminalcode}

\item[\texttt{touch}] updates the access and modification times of each FILE to 
    the current time.
   	\begin{terminalcode}
cami@bender:~/test$ ls -l
-rw-r--r-- 1 cami cami 0 2009-08-25 20:29 date.txt
cami@bender:~/test$ touch date.txt 
cami@bender:~/test$ ls -l
-rw-r--r-- 1 cami cami 0 2009-08-25 20:30 date.txt
    \end{terminalcode}

    It can be very useful to create a new empty file on the fly:
    \begin{terminalcode}
~/test$ ls
~/test$ touch emptyfile.txt
~/test$ ls
emptyfile.txt
    \end{terminalcode}


% add your own remarks here by reusing the existing examples

\end{description}

%=============================================================================
\section{Documentation with Latex}
%=============================================================================
\subsection{Introduction} 

In this section we explain some \LaTeX\xspace details and different formatting
commands.

Whenever you need to lookup a certain symbol for \LaTeX\xspace we suggest you to use
the online recognition tool \texttt{detexify} at \url{http://detexify.kirelabs.org/}.


%=============================================================================
\subsection{Common Commands}
\subsubsection{Sectioning}
Depening on the documentclass given in the very beginning of this file there
exist several sectioning levels:
\begin{enumerate}
	\item{} \verb$\section{NAME}$
	\item{} \verb$\subsection{NAME}$
	\item{} \verb$\subsubsection{NAME}$
	\item{} \verb$\paragraph{NAME}$
\end{enumerate}

\noindent To enforce \LaTeX to use a newline add a double slash \verb$\\$ at 
the end of a line.

\subsubsection{Schriftgrösse / -style}
\begin{tabular}{lll}                                                          
\verb$\rm$			& {\rm A normaler text}\\ 
\verb$\sl$ 			& {\sl An italic text}\\
\verb$\bf$ 			& {\bf A bold text}\\
\verb$\tiny$ 		& {\tiny A tiny ext}\\
\verb$\scriptsize$ 	& {\scriptsize A very, very small text}\\
\verb$\footnotesize$& {\footnotesize A very small text}\\
\verb$\small$ 		& {\small A small text}\\
\verb$\large$ 		& {\large A big text}\\
\verb$\Large$ 		& {\Large A bigger text}\\
\verb$\LARGE$ 		& {\LARGE An even bigger text}\\
\verb$\huge$ 	    & {\huge A huge text}\\
\verb$\Huge$ 	    & {\Huge A enormous huge text}\\
\verb$\emph$ 	    & \emph{An emphasized text} \\
\verb$\underline$ 	& \underline{An underlined text} \uline{and here using the ulem-package}\\
\verb$\texttt$ 		& \texttt{function goto(int a) { ... } }\\
\verb$\uuline$ 		& \uuline{A double unterstrichener text using the ulem-package} \\
\verb$\uwave$ 		& \uwave{A wavy unterstrichener text using the ulem-package} \\
\verb$\sout$ 	    & \sout{A crossed trough text using the ulem-package}\\
\verb$\xout$ 	    & \xout{A deleted text using the ulem-package}\\
\end{tabular}

\subsubsection{Notes}
To create a footnote use the \verb$\footnote{YOUR NOTE}$ 
command\footnote{\dots as you can see here.}. \\
If you want to put a remark at side of a page use \verb$\marginpar$.
\marginpar{This is a note at the border of the page.}

\subsubsection{Lists}
There exist several list types in \LaTeX. You start a list by adding a 
\verb$\being{LISTTYPE}$ and end it with an \verb$\end{LISTTYPE}$. A list item
is added with a \verb$\item$ between the \texttt{begin} and \texttt{end}.
\texttt{LISTTYPE} can be one of the following list:
\begin{itemize}
	\item \texttt{enumerate}
	\item \texttt{itemize}
	\item \texttt{description} with \verb$\item[topic]$
\end{itemize}
% By adding a \noindet the next line is not indented ;)
\noindent Note that you can nest lists if you want to.
\begin{enumerate}
	\item{e4} 	
		\begin{enumerate}
			\item{e4}   e5
			\item Lc4 d6
		\end{enumerate}
	\item Lc4 d6
\end{enumerate}

\subsubsection{Math, \LaTeX 's real strengths}
A much longer introduction, although still called a short math guide, is 
avaiable online at \url{ftp://ftp.ams.org/pub/tex/doc/amsmath/short-math-guide.pdf}.

\begin{description}

\item[Inline Mode]
Eiaquations with numeration with \verb$\begin{equation} FORMULA \end{equation}$:
\begin{equation} 
E_{kin} = \frac 1 2 m^2
\end{equation}

Equations without numeration with \verb$\begin{equation*} FORMULA \end{equation*}$:
\begin{equation*} 
E_{kin} = \frac 1 2 m^2
\end{equation*}

Shortcut using \verb$\[ FORMULA \]$:
\[ -\frac{\hbar}{2m}\Delta\Phi(\vec r) + V(\vec r)\Phi(\vec r) = E\Phi(\vec r) \]

Inline mode with \verb|$ FORMULA $| displays as 
$\int_\infty^\infty |\psi(x)^2|\mathrm{dx} = 1$.

\item[Parenthesis]
\[ \Biggl( \biggl( \Bigl( \bigl( ( ) \bigr) \Bigr) \biggr) \Biggr) \]

\item[Spaces]
\begin{tabular}[t]{lll}
Small spaces        & \verb$\_$     & $ y=x^{2} \_ y'=2x \_  y''=2 $ \\
Middle sized spaces & \verb$\quad$  & $ y=x^{2} \quad y'=2x \quad  y''=2 $ \\
Big spaces          & \verb$\qquad$ & $ y=x^{2} \qquad y'=2x \qquad  y''=2 $
\end{tabular}

\item[Indices and Powers]
\[ a_i, x^{n+1} \qquad a_{ij} + b_{ij} = p_{ij} \qquad 
    \text{\ldots and nested} \qquad
    a_{x_ij} = n_{x^{2^b_n}} \]

\item[Fractions]
\[  \frac{Zaehler}{Nenner} \qquad 
    \frac{a}{b} + \frac{c}{b} = \frac{ a+c}{b} \qquad
    \frac{\frac{\frac{a}{b}}{c}}{d} \qquad
    \frac {{n+1 \choose k/2}} {5!} \]

\noindent In the simple math environment two FORMULAdifferent sized fractions can be 
used; the small fractions $\frac{1}{2}$ or the normal sized $\dfrac{1}{x}$.

\item[Roots]    
\[  \sqrt[\text{root depth}]{\text{root term}} \qquad
    \sqrt{x+y-z}, \sqrt[5]{4+x} \]
                
\item[Functions]
\[ f : \mathbb{N} \to \mathbb{R} \qquad f : x \mapsto x^2\]
Mathematical functions are writtein explicitely written in normal text not
math mode text:
\[ \sin(x) = \text{sin}(x) \text{ {\bf and not }} sin(x)\]

\item[Varia]
\[ \left( \sqrt{\frac{A^C}{B_y}} +       \sum_{i=1}^N a_i\right) \]
\[ A \stackrel {\lambda_a} {\longrightarrow} B \]
\[ \int \! \! \int z\,dx \, dy \quad \mbox {\bf not} \quad \int\int z dx dy \]
\[ \int \! \! \int z\,dx \, dy \quad \mbox {\bf not} \quad \int\int z dx dy \]
                                
{\Large
\[ \Leftarrow \  \Leftrightarrow \  \Longleftrightarrow \  \Rightarrow \_ 
    \Uparrow \  \Updownarrow \  \Downarrow      \]}
                                
\[\bigcap \cap \sum \int_0^{2\pi} \vec{a} \dot{a} \ddot{a} a^{\prime \prime} \]


\item[Matrices]
\[ \det A = \| a_{ik} \| =
\left| \begin{array}{ccccc}
    a_{11} & a_{12} & a_{13} &  \cdots & a_{1n} \\
    a_{21} & a_{22} & a_{23} &  \cdots & a_{2n} \\
    \vdots  & \vdots  & \vdots  & \ddots & \vdots\\
    a_{n1} & a_{n2} & a_{n3} &  \cdots & a_{nn} \\
\end{array} \right| . \]

\end{description}


%=============================================================================

\section{Ruby Programming}
%=============================================================================
Ruby is an open source programming language. It can also simply be used as a calculator. In the following chapter, you'll find a few helpful commands.

\subsection{Commands}
\begin{description}

\item[\texttt{irb}]open ruby

\item[\texttt{quit}]quit irb

\item[\texttt{puts}]basic command to print something out in ruby

\item[\texttt{math}]module for mathematics

\item[\texttt{def}]start of a definition

\item[\texttt{end}]tells ruby that the definition or the class is completed

\item[\texttt{class}]defines a new class

\end{description}

\end{document}
