\documentclass[10pt,a4paper]{scrartcl}
%=============================================================================
\usepackage[utf8]{inputenc}  
\usepackage[normalem]{ulem} % \emph should italicize, not underline
\usepackage{alltt}
\usepackage{amsmath}
\usepackage{amssymb}
%\usepackage{bold-extra}
\usepackage{cite}
\usepackage{graphicx}
\usepackage{ifthen}
\usepackage{subfigure}
\usepackage{xspace}

% source code formatting
\usepackage{listings}
	% global settings for source code listing pacakage
\lstset{
    basicstyle=\ttfamily,
    showspaces=false,
    showstringspaces=false,
    captionpos=b, 
    columns=fullflexible}
	% define the listing shortcuts for java and python
\lstnewenvironment{terminalcode}[1][]{\lstset{language=bash,#1}}{} 

%----------------------------------------------------------------------------

% enabled links in pdf, but paint them normal in black
\usepackage[pdftex=true, colorlinks=true, urlcolor=black, 
			linkcolor=black,pagecolor=black,citecolor=black,
			bookmarks=true]{hyperref}

%=============================================================================

\date{HS 2009 University Bern}
\author{Christa Biberstein}
\title{Programming Introduction}
\begin{document}
\maketitle
%=============================================================================
\section{Terminal}
%=============================================================================
\subsection{Introduction}
\begin{terminalcode}
> uname -mns
  Darwin imac.local i386
  Report bugs to <bug-coreutils@gnu.org>.
> uname -mns
  Darwin mbkp.local i386
> ssh anker.unibe.ch
  user@bender.unibe.ch's password: 
> uname
  Linux
> uname -mon
  bender x86_64 GNU/Linux
> uname --help
  Usage: uname [OPTION]...
  Print certain system information.  With no OPTION, same as -s.
  
    -a, --all                print all information, in the following order,
                               except omit -p and -i if unknown:
    -s, --kernel-name        print the kernel name
    -n, --nodename           print the network node hostname
    -r, --kernel-release     print the kernel release
    -v, --kernel-version     print the kernel version
    -m, --machine            print the machine hardware name
    -p, --processor          print the processor type or "unknown"
    -i, --hardware-platform  print the hardware platform or "unknown"
    -o, --operating-system   print the operating system
        --help     display this help and exit
        --version  output version information and exit
\end{terminalcode}

%=============================================================================
\subsection{Commands}
\begin{description}

\item[\bf{pwd}] print working directory
    \begin{terminalcode}
[kurs16@kollwitz documentation]$ pwd
/home/kurs16/Documents/documentation
\end{terminalcode}


\item[\bf{ls}]list directory contents
    \begin{terminalcode}
[kurs16@kollwitz documentation]$ ls
documentation.aux  documentation.log  documentation.out  documentation.pdf  documentation.tex  Makefile
\end{terminalcode}


\item[\bf{cd}] change the working directory
    \begin{terminalcode}
1[kurs16@kollwitz ~]$ cd Documents/
 [kurs16@kollwitz ~/Documents]$ 
2[kurs16@kollwitz documentation]$ cd ..
 [kurs16@kollwitz ~/Documents]$ 
3 cd ~   change to home
 [kurs16@kollwitz documentation]$ cd ~
 [kurs16@kollwitz ~]$ 
\end{terminalcode}


\item[\bf{:w}] save

\item[\bf{q}] quit

\item[\bf{whoami}] print effective userid
    \begin{terminalcode}
[kurs16@kollwitz lessons]$ whoami
kurs16
    \end{terminalcode}

\item[\bf{Ctrl-C}] if the command is frozen kill it by typing Ctrl-C.

\item[\bf{ls -GlF}] list the contents with more information 
    \begin{terminalcode}
[kurs16@kollwitz lessons]$ ls -GlF
total 24
-rw-r--r--  1 kurs16 4086 2009-09-08 09:10 00 terminal introduction.txt
drwxr-xr-x  3 kurs16 4096 2009-09-08 09:10 01 vim introduction/
drwxr-xr-x  3 kurs16 4096 2009-09-07 09:16 02 documentation/
drwxr-xr-x 22 kurs16 4096 2009-09-07 09:16 03 terminal basic commands/
drwxr-xr-x  6 kurs16 4096 2009-09-07 09:16 04 terminal extension/
drwxr-xr-x  7 kurs16 4096 2009-09-08 09:10 05 ruby/
    \end{terminalcode}

\item[\bf{whatis COMMAND}] more information about a COMMAND
    \begin{terminalcode}
[kurs16@kollwitz documentation]$ whatis ls
ls                   (1)  - list directory contents
also "man COMMAND", "info COMMAND", "help COMMAND", "COMMAND --help"    
    \end{terminalcode}

\item[\bf{TAB}] tab auto-completion
    \begin{terminalcode}
[kurs16@kollwitz ~]$ cd Doc "TAB"
[kurs16@kollwitz ~]$ cd Documents/ 
    \end{terminalcode}

\item[\bf{COMMAND | less}] useful if the output doesn't fit on the screen
    \begin{terminalcode}
Now you can scroll with the arrow keys up and down. Again use "q" to exit.
    \end{terminalcode}

\item[\bf{svn up}] update changes

\item[\bf{rm}]
    \begin{terminalcode}
cami@bender:~/test$ ls
todelete.txt
cami@bender:~/test$ rm todelete.txt 
cami@bender:~/test$ ls
    \end{terminalcode}

\item[\bf{touch}] updates the access and modification times of each FILE to 
    the current time.
        \begin{terminalcode}
cami@bender:~/test$ ls -l
-rw-r--r-- 1 cami cami 0 2009-08-25 20:29 date.txt
cami@bender:~/test$ touch date.txt 
cami@bender:~/test$ ls -l
-rw-r--r-- 1 cami cami 0 2009-08-25 20:30 date.txt
    \end{terminalcode}

    It can be very useful to create a new empty file on the fly:
    \begin{terminalcode}
~/test$ ls
~/test$ touch emptyfile.txt
~/test$ ls
emptyfile.txt
    \end{terminalcode}




\item[\large{How to use vim:}]

\item[\bf{vim DOCUMENT}] Vi IMproved, a programmers text editor (opens the document)

\item[\bf{h}] move left

\item[\bf{j}] move down

\item[\bf{k}] move up

\item[\bf{l}] move right

\item[\bf{:q!}] exit vim and trash all changes

\item[\bf{:wq}] exit vim and save all changes

\item[\bf{:w}] save all changes

\item[\bf{x}] To delete the character at the cursor type

\item[\bf{i}] type inserted text (insert before the cursor)

\item[\bf{A}] type appended text (append after the line)

\item[\bf{ESC}]  Pressing <ESC> will place you in Normal mode or will cancel
      an unwanted and partially completed command.

\item[\bf{d}] to delete

\item[\bf{dw}] To delete from the cursor up to the next word

\item[\bf{d\$}] To delete from the cursor to the end of a line

\item[\bf{dd}] To delete a whole line

\item[\bf{w}] until the start of the next word, EXCLUDING its first character

\item[\bf{e}] to the end of the current word, INCLUDING the last character

\item[\bf{\$}] to the end of the line, INCLUDING the last caracter

\item[\bf{0}] to move to the start of the line

\item[\bf{No.Motion}] Typing a number before a motion repeats it that many times

\item[\bf{u}] to undo the last command executed/ previous action

\item[\bf{U}] to undo all the changes on a line

\item[\bf{Ctrl-R}] to undo the undo's

      operator - is what to do, such as  d  for delete
      [number] - is an optional count to repeat the motion
      motion   - moves over the text to operate on, such as  w (word),
                  \$ (to the end of line), etc.

\item[\bf{p}] to put previously deleted text after the cursor

\item[\bf{r}] To replace the character under the cursor, type   r   and then the
     character you want to have there.

\item[\bf{c}]  The change operator allows you to change from the cursor to where the
     motion takes you.  eg. Type  ce  to change from the cursor to the end of
     the word,  c\$  to change to the end of a line.

\item[\bf{Ctrl-G}] displays your location in the file and the file status.

\item[\bf{G}] moves to the end of the file

\item[\bf{No.G}] moves to that line number

\item[\bf{gg}] moves to the first line

\item[\bf{/}] followed by a phrase searches FORWARD for the phrase

\item[\bf{?}] followed by a phrase searches BACKWARD for the phrase

\item[\bf{n}] to find the next occurrence in the same direction after a search

\item[\bf{N}] to find the next occurrence in the opposite direction after a search

\item[\bf{Ctrl-O}] takes you back to older positions

\item[\bf{Ctrl-I}] takes you back to newer positions

\item[\bf{\%}] while the cursor is on a (,),[,],{, or } goes to its match

\item[\bf{:s/old/new}] to substitute new for the first old in a line

\item[\bf{:s/old/new/g}] To substitute new for all 'old's on a line type 

\item[\bf{:\#,\#s/old/new/g}] To substitute phrases between two line \#'s type

\item[\bf{:\%s/old/new/g}] To substitute all occurrences in the file type

\item[\bf{:\%s/old/new/gc}] To ask for confirmation each time add 'c'

\item[\bf{:!}] Type  :!  followed by an external command to execute that command

Examples: (MS-DOS)         (Unix)
          :!dir            :!ls            -  shows a directory listing.
          :!del FILENAME   :!rm FILENAME   -  removes file FILENAME.

\item[\bf{:w FILENAME}] writes the current Vim file to disk with name FILENAME

\item[\bf{v  motion  :w FILENAME}]  saves the Visually selected lines in file FILENAME

\item[\bf{:r FILENAME}] retrieves disk file FILENAME and puts it below the
      cursor position.

\item[\bf{:r !dir}] reads the output of the dir command and puts it below the
      cursor position.

\item[\bf{o}] to open a line BELOW the cursor and start Insert mode.

\item[\bf{O}] to open a line ABOVE the cursor and atart Insert mode.

\item[\bf{a}] to insert text AFTER the cursor.

\item[\bf{A}] to insert text after the end of the line.

\item[\bf{y}] operator yanks (copies) text

\item[\bf{R}] Typing a capital  R  enters Replace mode until  <ESC>  is pressed.

\item[\bf{Typing ":set xxx" sets the option "xxx".}] 
       Some options are:
        'ic' 'ignorecase'       ignore upper/lower case when searching
        'is' 'incsearch'        show partial matches for a search phrase
        'hls' 'hlsearch'        highlight all matching phrases
       You can either use the long or the short option name.

\item[\bf{:set noic}] Prepend "no" to switch an option off

\item[\bf{help}]
  1. Type  :help  or press <F1> or <Help>  to open a help window.

  2. Type  :help cmd  to find help on  cmd .

  3. Type  CTRL-W CTRL-W  to jump to another window

  4. Type  :q  to close the help window

  5. Create a vimrc startup script to keep your preferred settings.

  6. When typing a  :  command, press CTRL-D to see possible completions.
     Press <TAB> to use one completion.



% add your own remarks here by reusing the existing examples

\end{description}

%=============================================================================
\section{Documentation with Latex}
%=============================================================================
\subsection{Introduction} 

In this section we explain some \LaTeX\xspace details and different formatting
commands.

Whenever you need to lookup a certain symbol for \LaTeX\xspace we suggest you to use
the online recognition tool \texttt{detexify} at \url{http://detexify.kirelabs.org/}.


%=============================================================================
\subsection{Common Commands}
\subsubsection{Sectioning}
Depening on the documentclass given in the very beginning of this file there
exist several sectioning levels:
\begin{enumerate}
	\item{} \verb$\section{NAME}$
	\item{} \verb$\subsection{NAME}$
	\item{} \verb$\subsubsection{NAME}$
	\item{} \verb$\paragraph{NAME}$
\end{enumerate}

\noindent To enforce \LaTeX to use a newline add a double slash \verb$\\$ at 
the end of a line.

\subsubsection{Schriftgrösse / -style}
\begin{tabular}{lll}                                                          
\verb$\rm$			& {\rm A normaler text}\\ 
\verb$\sl$ 			& {\sl An italic text}\\
\verb$\bf$ 			& {\bf A bold text}\\
\verb$\tiny$ 		& {\tiny A tiny ext}\\
\verb$\scriptsize$ 	& {\scriptsize A very, very small text}\\
\verb$\footnotesize$& {\footnotesize A very small text}\\
\verb$\small$ 		& {\small A small text}\\
\verb$\large$ 		& {\large A big text}\\
\verb$\Large$ 		& {\Large A bigger text}\\
\verb$\LARGE$ 		& {\LARGE An even bigger text}\\
\verb$\huge$ 	    & {\huge A huge text}\\
\verb$\Huge$ 	    & {\Huge A enormous huge text}\\
\verb$\emph$ 	    & \emph{An emphasized text} \\
\verb$\underline$ 	& \underline{An underlined text} \uline{and here using the ulem-package}\\
\verb$\texttt$ 		& \texttt{function goto(int a) { ... } }\\
\verb$\uuline$ 		& \uuline{A double unterstrichener text using the ulem-package} \\
\verb$\uwave$ 		& \uwave{A wavy unterstrichener text using the ulem-package} \\
\verb$\sout$ 	    & \sout{A crossed trough text using the ulem-package}\\
\verb$\xout$ 	    & \xout{A deleted text using the ulem-package}\\
\end{tabular}

\subsubsection{Notes}
To create a footnote use the \verb$\footnote{YOUR NOTE}$ 
command\footnote{\dots as you can see here.}. \\
If you want to put a remark at side of a page use \verb$\marginpar$.
\marginpar{This is a note at the border of the page.}

\subsubsection{Lists}
There exist several list types in \LaTeX. You start a list by adding a 
\verb$\being{LISTTYPE}$ and end it with an \verb$\end{LISTTYPE}$. A list item
is added with a \verb$\item$ between the \texttt{begin} and \texttt{end}.
\texttt{LISTTYPE} can be one of the following list:
\begin{itemize}
	\item \texttt{enumerate}
	\item \texttt{itemize}
	\item \texttt{description} with \verb$\item[topic]$
\end{itemize}
% By adding a \noindet the next line is not indented ;)
\noindent Note that you can nest lists if you want to.
\begin{enumerate}
	\item{e4} 	
		\begin{enumerate}
			\item{e4}   e5
			\item Lc4 d6
		\end{enumerate}
	\item Lc4 d6
\end{enumerate}

\subsubsection{Math, \LaTeX 's real strengths}
A much longer introduction, although still called a short math guide, is 
avaiable online at \url{ftp://ftp.ams.org/pub/tex/doc/amsmath/short-math-guide.pdf}.

\begin{description}

\item[Inline Mode]
Equations with numeration with \verb$\begin{equation} FORMULA \end{equation}$:
\begin{equation} 
E_{kin} = \frac 1 2 m v^2
\end{equation}

Equations without numeration with \verb$\begin{equation*} FORMULA \end{equation*}$:
\begin{equation*} 
E_{kin} = \frac 1 2 m v^2
\end{equation*}

Shortcut using \verb$\[ FORMULA \]$:
\[ -\frac{\hbar}{2m}\Delta\Phi(\vec r) + V(\vec r)\Phi(\vec r) = E\Phi(\vec r) \]

Inline mode with \verb|$ FORMULA $| displays as 
$\int_\infty^\infty |\psi(x)^2|\mathrm{dx} = 1$.

\item[Parenthesis]
\[ \Biggl( \biggl( \Bigl( \bigl( ( ) \bigr) \Bigr) \biggr) \Biggr) \]

\item[Spaces]
\begin{tabular}[t]{lll}
Small spaces        & \verb$\_$     & $ y=x^{2} \_ y'=2x \_  y''=2 $ \\
Middle sized spaces & \verb$\quad$  & $ y=x^{2} \quad y'=2x \quad  y''=2 $ \\
Big spaces          & \verb$\qquad$ & $ y=x^{2} \qquad y'=2x \qquad  y''=2 $
\end{tabular}

\item[Indices and Powers]
\[ a_i, x^{n+1} \qquad a_{ij} + b_{ij} = p_{ij} \qquad 
    \text{\ldots and nested} \qquad
    a_{x_ij} = n_{x^{2^b_n}} \]

\item[Fractions]
\[  \frac{Zaehler}{Nenner} \qquad 
    \frac{a}{b} + \frac{c}{b} = \frac{ a+c}{b} \qquad
    \frac{\frac{\frac{a}{b}}{c}}{d} \qquad
    \frac {{n+1 \choose k/2}} {5!} \]

\noindent In the simple math environment two FORMULAdifferent sized fractions can be 
used; the small fractions $\frac{1}{2}$ or the normal sized $\dfrac{1}{x}$.

\item[Roots]    
\[  \sqrt[\text{root depth}]{\text{root term}} \qquad
    \sqrt{x+y-z}, \sqrt[5]{4+x} \]
                
\item[Functions]
\[ f : \mathbb{N} \to \mathbb{R} \qquad f : x \mapsto x^2\]
Mathematical functions are writtein explicitely written in normal text not
math mode text:
\[ \sin(x) = \text{sin}(x) \text{ {\bf and not }} sin(x)\]

\item[Varia]
\[ \left( \sqrt{\frac{A^C}{B_y}} +       \sum_{i=1}^N a_i\right) \]
\[ A \stackrel {\lambda_a} {\longrightarrow} B \]
\[ \int \! \! \int z\,dx \, dy \quad \mbox {\bf not} \quad \int\int z dx dy \]
\[ \int \! \! \int z\,dx \, dy \quad \mbox {\bf not} \quad \int\int z dx dy \]
                                
{\Large
\[ \Leftarrow \  \Leftrightarrow \  \Longleftrightarrow \  \Rightarrow \_ 
    \Uparrow \  \Updownarrow \  \Downarrow      \]}
                                
\[\bigcap \cap \sum \int_0^{2\pi} \vec{a} \dot{a} \ddot{a} a^{\prime \prime} \]


\item[Matrices]
\[ \det A = \| a_{ik} \| =
\left| \begin{array}{ccccc}
    a_{11} & a_{12} & a_{13} &  \cdots & a_{1n} \\
    a_{21} & a_{22} & a_{23} &  \cdots & a_{2n} \\
    \vdots  & \vdots  & \vdots  & \ddots & \vdots\\
    a_{n1} & a_{n2} & a_{n3} &  \cdots & a_{nn} \\
\end{array} \right| . \]

\end{description}

%=============================================================================
\section{Ruby Programming}
%=============================================================================


\end{document}
