%=============================================================================
% This document shows the basic of LaTeX. If you want to learn more we suggest
% reading the not so short latex introduction:
%                   http://tobi.oetiker.ch/lshort/lshort.pdf
%=============================================================================

% Lines starting with a percent sign are comments, and thus ignored and not
% shown in the final output. 

% Although you can write normal text in Latex, special commands allow
% formatting or say where a new section start. Such commands start with a
% backslash (\). For instance to render text bold, we simply write \bf{foo},
% now 'foo' is going to shown in bold style.

% The rest of this document is stuffed with explanations and examples. Try to
% copy existing code whenever possible and compile the document as often as
% possible. Latex is horrible at giving you clues about the errors you made.

%=============================================================================
% This line specifies the type of the document
\documentclass[10pt,a4paper]{scrartcl}
% possbile options are the fontsize (here "10pt")
%   the paper size (here "a4paper")
%   the pagetype (not present: "twopage" or "oneside")

% there exist various kind of documentclasses
%   article / scrticle
%   report / scrreport
%   book / scrbook
%   slides
%   letter / scrlttr2

%=============================================================================
% In this section follows the inclusion of different additional packages.
% Each package contains a certain set of additional features like special
% formatting language settings or for instance, tables.

% Make sure the input is handled correctly
\usepackage[utf8]{inputenc}

% This packages enables fancy text formatting / underlining ...
\usepackage[normalem]{ulem}

% Latex math support
% ftp://ftp.ams.org/pub/tex/doc/amsmath/short-math-guide.pdf
\usepackage{amsmath}
% Add special math symbols
\usepackage{amssymb}

% Enable citations of scientific documents
\usepackage{cite}

% Enable \includegraphics for easy insertion of graphics into a latex doc
\usepackage{graphicx}

% Enable adding several figures in a row (not used in this doc, yet)
\usepackage{subfigure}

% Add support for \xspace a small space
\usepackage{xspace}

%----------------------------------------------------------------------------
% This section sets the properties for listings. Listings are used to display
% verbatim (unformatted and with a fixed width font style) text like source
% code or terminal sessions.  Source code formatting can be set in detail:
\usepackage{listings}
	% Global / primary settings for source code listing package
\lstset{
    basicstyle=\ttfamily,
    showspaces=false,
    showstringspaces=false,
    captionpos=b, 
    columns=fullflexible}

% Define the listing shortcuts for terminal code (bash)
\lstnewenvironment{terminalcode}[1][]{\lstset{language=bash,#1}}{} 

%----------------------------------------------------------------------------

% The hyperref packages enables links in the outputed pdf enabled links in
% pdf, but paint them normal in black
\usepackage[pdftex=true, colorlinks=true, urlcolor=black, 
			linkcolor=black,pagecolor=black,citecolor=black,
			bookmarks=true]{hyperref}

%=============================================================================

% Here the properties for the title of the document are set
\date{HS 2010 University Bern}
\author{PUT YOUR NAME HERE}
\title{Programming Introduction}

%=============================================================================
% Past this line the real content of the file starts with \begin{document} and
% ends with \end{document} at the very end of this file
\begin{document}

% put the title page here
\maketitle

% put the table of contents here
\tableofcontents

% add a new page
\newpage

%=============================================================================
% start a new section named "Terminal".
\section{Terminal}
%=============================================================================
% start a new subsection named "Introduction"
\subsection{Introduction}

% add a verbatim code block for the example bash session
% we start again with \begin{terminalcode} and end with \end{terminalcode}
% everything in between is printed with a fixedverbatimverbatim with font without formatting.
\begin{terminalcode}
> uname -mns
  Darwin imac.local i386
  Report bugs to <bug-coreutils@gnu.org>.
> uname -mns
  Darwin mbkp.local i386
> ssh anker.unibe.ch
  user@bender.unibe.ch's password: 
> uname
  Linux
> uname -mon
  bender x86_64 GNU/Linux
> uname --help
  Usage: uname [OPTION]...
  Print certain system information.  With no OPTION, same as -s.
  
    -a, --all                print all information, in the following order,
                               except omit -p and -i if unknown:
    -s, --kernel-name        print the kernel name
    -n, --nodename           print the network node hostname
    -r, --kernel-release     print the kernel release
    -v, --kernel-version     print the kernel version
    -m, --machine            print the machine hardware name
    -p, --processor          print the processor type or "unknown"
    -i, --hardware-platform  print the hardware platform or "unknown"
    -o, --operating-system   print the operating system
        --help     display this help and exit
        --version  output version information and exit
\end{terminalcode}
% the previous line ended a verbatim code section (note the \end)

%=============================================================================
% again starting a new subsection
\subsection{Commands}
% add a description list. A description list is again an certain special
% block (called environment in latex terms) formatting the enclosed text
% The ouput will look something like this:
%   item_name text text text text
%             text text text text
%   item_name text text text text
%             text text text text
%
%
\begin{description}

% Here we add a new item to the descrption list
% Each item is added with a special command \item[ITEM_NAME] where ITEM_NAME
% is text you add.
% In this example ITEM_NAME is "\texttt{rm}". \texttt is a formatting commondd
% displaying the parenthesized text in verbatim and a fixed sized font
\item[\texttt{rm}] removes a file or a directory
% now we add a code section into the item
    \begin{terminalcode}
cami@bender:~/test$ ls
todelete.txt
cami@bender:~/test$ rm todelete.txt 
cami@bender:~/test$ ls
    \end{terminalcode}
% the code block ended with the previous line

% again anothe item followed by some descriptive text
\item[\texttt{touch}] updates the access and modification times of each FILE to 
    the current time.
   	\begin{terminalcode}
cami@bender:~/test$ ls -l
-rw-r--r-- 1 cami cami 0 2009-08-25 20:29 date.txt
cami@bender:~/test$ touch date.txt 
cami@bender:~/test$ ls -l
-rw-r--r-- 1 cami cami 0 2009-08-25 20:30 date.txt
    \end{terminalcode}

    It can be very useful to create a new empty file on the fly:
    \begin{terminalcode}
~/test$ ls
~/test$ touch emptyfile.txt
~/test$ ls
emptyfile.txt
    \end{terminalcode}


% add your own remarks here by reusing the existing examples

% ending a description list
\end{description}


%=============================================================================
% starting a new section
\section{Documentation with Latex}
%=============================================================================
\subsection{Introduction} 

% \LaTeX displays the latex logo, \xpsace a tiny space
In this section we explain some \LaTeX\xspace details and different formatting
commands.

Whenever you need to lookup a certain symbol for \LaTeX\xspace we suggest you
to use the online recognition tool \texttt{detexify} at
\url{http://detexify.kirelabs.org/}.


%=============================================================================
\subsection{Common Commands}
% ----------------------------------------------------------------------------
\subsubsection{Sectioning}
Depening on the documentclass given in the very beginning of this file there
exist several sectioning levels:
% here we begin a list again but of another type named "enumerate":
\begin{enumerate}
\item{} \verb$\section{NAME}$
\item{} \verb$\subsection{NAME}$
\item{} \verb$\subsubsection{NAME}$
\item{} \verb$\paragraph{NAME}$
\end{enumerate}

% the \nondent causes the folowing block not to be indented ( = having a 
% certain space at the beginning)
\noindent To enforce \LaTeX to use a newline add a double slash \verb$\\$ at 
the end of a line.

% ----------------------------------------------------------------------------
\subsubsection{Font size and style}
% here we begin a new table having three columns "{lll}". Each "l" represents
% a column being left aligned. Other possible values would be "c" (centered)
% or "r" (right aligned):
\begin{tabular}{lll}
%                   each column is separated with an ampersand "&"
\verb$\rm$			& \rm{A normal text}\\ 
%                                         each new row is ended with "\\"
\verb$\sl$ 			& \sl{An italic text}\\
% \verb$$ does the same as \texttt{} but also allows commands to be displayed
% in verbatim
\verb$\bf$ 			& \bf{A bold text}\\
\verb$\tiny$ 		& \tiny{A tiny ext}\\
\verb$\scriptsize$ 	& \scriptsize{A very, very small text}\\
\verb$\footnotesize$& \footnotesize{A very small text}\\
\verb$\small$ 		& \small{A small text}\\
\verb$\large$ 		& \large{A big text}\\
\verb$\Large$ 		& \Large{A bigger text}\\
\verb$\LARGE$ 		& \LARGE{An even bigger text}\\
\verb$\huge$ 	    & \huge{A huge text}\\
\verb$\Huge$ 	    & \Huge{An enormous huge text}\\
\verb$\emph$ 	    & \emph{An emphasized text} \\
\verb$\underline$ 	& \underline{An underlined text} \uline{and here using the ulem-package}\\
\verb$\texttt$ 		& \texttt{function goto(int a) { ... } }\\
\verb$\uuline$ 		& \uuline{A double unterstrichener text using the ulem-package} \\
\verb$\uwave$ 		& \uwave{A wavy unterstrichener text using the ulem-package} \\
\verb$\sout$ 	    & \sout{A crossed trough text using the ulem-package}\\
\verb$\xout$ 	    & \xout{A deleted text using the ulem-package}\\
\end{tabular}

% ----------------------------------------------------------------------------
\subsubsection{Notes}
To create a footnote use the \verb$\footnote{YOUR NOTE}$
command\footnote{\dots as you can see here.}. \\ If you want to put a remark
at side of a page use \verb$\marginpar$.  \marginpar{This is a note at the
border of the page.}

% ----------------------------------------------------------------------------
\subsubsection{Lists}
There exist several list types in \LaTeX. You start a list by adding a
\verb$\being{LISTTYPE}$ and end it with an \verb$\end{LISTTYPE}$. A list item
is added with a \verb$\item$ between the \texttt{begin} and \texttt{end}.
\texttt{LISTTYPE} can be one of the following list:

\begin{itemize}
\item \texttt{enumerate}
\item \texttt{itemize}
\item \texttt{description} with \verb$\item[topic]$
\end{itemize}
% By adding a \noindet the next line is not indented ;)
\noindent Note that you can nest lists if you want to.

\begin{enumerate}
% start a new item named "e4"
\item{e4} 	
% add a list into the item named "e4"
	\begin{enumerate}
% add items to the nested list
	\item{e4}   e5
	\item Lc4 d6
% end the nested list
	\end{enumerate}
\item Lc4 d6
\end{enumerate}


% ----------------------------------------------------------------------------
\subsubsection{Math, \LaTeX 's real strengths}
A much longer introduction, although still called a short math guide, is 
avaiable online at \url{ftp://ftp.ams.org/pub/tex/doc/amsmath/short-math-guide.pdf}.

\begin{description}

\item[Inline Mode]
Equations with numeration with \verb$\begin{equation} FORMULA \end{equation}$:
\begin{equation} 
E_{kin} = \frac 1 2 m v^2
\end{equation}

Equations without numeration with \verb$\begin{equation*} FORMULA \end{equation*}$:
\begin{equation*} 
E_{kin} = \frac 1 2 m v^2
\end{equation*}

Shortcut using \verb$\[ FORMULA \]$:
\[ -\frac{\hbar}{2m}\Delta\Phi(\vec r) + V(\vec r)\Phi(\vec r) = E\Phi(\vec r) \]

Inline mode with \verb|$ FORMULA $| displays as 
$\int_\infty^\infty |\psi(x)^2|\mathrm{dx} = 1$.

\item[Parenthesis]
\[ \Biggl( \biggl( \Bigl( \bigl( ( ) \bigr) \Bigr) \biggr) \Biggr) \]

\item[Spaces]
\begin{tabular}[t]{lll}
Small spaces        & \verb$\_$     & $ y=x^{2} \_ y'=2x \_  y''=2 $ \\
Middle sized spaces & \verb$\quad$  & $ y=x^{2} \quad y'=2x \quad  y''=2 $ \\
Big spaces          & \verb$\qquad$ & $ y=x^{2} \qquad y'=2x \qquad  y''=2 $
\end{tabular}

\item[Indices and Powers]
\[ a_i, x^{n+1} \qquad a_{ij} + b_{ij} = p_{ij} \qquad 
    \text{\ldots and nested} \qquad
    a_{x_ij} = n_{x^{2^b_n}} \]

\item[Fractions]
\[  \frac{Zaehler}{Nenner} \qquad 
    \frac{a}{b} + \frac{c}{b} = \frac{ a+c}{b} \qquad
    \frac{\frac{\frac{a}{b}}{c}}{d} \qquad
    \frac {{n+1 \choose k/2}} {5!} \]

\noindent In the simple math environment two FORMULAdifferent sized fractions can be 
used; the small fractions $\frac{1}{2}$ or the normal sized $\dfrac{1}{x}$.

\item[Roots]	
\[  \sqrt[\text{root depth}]{\text{root term}} \qquad
    \sqrt{x+y-z}, \sqrt[5]{4+x} \]
		
\item[Functions]
\[ f : \mathbb{N} \to \mathbb{R} \qquad f : x \mapsto x^2\]
Mathematical functions are writtein explicitely written in normal text not
math mode text:
\[ \sin(x) = \text{sin}(x) \text{ {\bf and not }} sin(x)\]

\item[Varia]
\[ \left( \sqrt{\frac{A^C}{B_y}} +	 \sum_{i=1}^N a_i\right) \]
\[ A \stackrel {\lambda_a} {\longrightarrow} B \]
\[ \int \! \! \int z\,dx \, dy \quad \mbox {\bf not} \quad \int\int z dx dy \]
\[ \int \! \! \int z\,dx \, dy \quad \mbox {\bf not} \quad \int\int z dx dy \]
				
{\Large
\[ \Leftarrow \  \Leftrightarrow \  \Longleftrightarrow \  \Rightarrow \_ 
    \Uparrow \  \Updownarrow \  \Downarrow	\]}
				
\[\bigcap \cap \sum \int_0^{2\pi} \vec{a} \dot{a} \ddot{a} a^{\prime \prime} \]


\item[Matrices]
\[ \det A = \| a_{ik} \| =
\left| \begin{array}{ccccc}
    a_{11} & a_{12} & a_{13} &  \cdots & a_{1n} \\
    a_{21} & a_{22} & a_{23} &  \cdots & a_{2n} \\
    \vdots  & \vdots  & \vdots  & \ddots & \vdots\\
    a_{n1} & a_{n2} & a_{n3} &  \cdots & a_{nn} \\
\end{array} \right| . \]

\end{description}


%=============================================================================
\section{Ruby Programming}
%=============================================================================


\end{document}
